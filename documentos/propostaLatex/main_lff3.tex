%% main_ppgco_ufu.tex v1.0, Lásaro Camargos e Denise Guliato
% adaptado de modeloABNT2.tex, v1.0 athila 
% ------------------------------------------------------------------------
% ------------------------------------------------------------------------
% eesc: Modelo de Trabalho Acadêmico (tese de doutorado, dissertação de
% mestrado e trabalhos monográficos em geral) em conformidade com 
% ABNT NBR 14724:2011. Esta classe estende as funcionalidades da classe
% abnTeX2 elaborada de forma a adequar os parâmetros exigidos pelas 
% normas USP e do departamento de elétrica da Escola de Engenharia 
% de São Carlos - USP.
% ------------------------------------------------------------------------
% ------------------------------------------------------------------------

% ------------------------------------------------------------------------
% Opções:
% 	tesedr:     Formata documento para tese de doutorado
%	qualidr:    Formata documento para qualificação de doutorado
% 	dissertmst: Formata documento para dissertação de mestrado
% 	qualimst:   Formata documento para qualificação de mestrado
% ------------------------------------------------------------------------
\documentclass[dissertmst]{ppgco}
%Não altere o comando seguinte. O título de seu trabalho será especificado mais adiante.
\title{Proposta do projeto final de Data Science - Violência Armada e suas tendências futuras nos EUA.  
}

% ---
% PACOTES
% ---

% ---
% Pacotes fundamentais 
% ---
\usepackage{cmap}				% Mapear caracteres especiais no PDF
\usepackage{lmodern}				% Usa a fonte Latin Modern			
\usepackage{makeidx}            	% Cria o indice
\usepackage{hyperref}  			% Controla a formação do índice
\usepackage{lastpage}			% Usado pela Ficha catalográfica
\usepackage{indentfirst}			% Indenta o primeiro parágrafo de cada seção.
\usepackage{nomencl} 			% Lista de simbolos
\usepackage{graphicx}			% Inclusão de gráficos
% ---

% ---
% Pacotes adicionais, usados apenas no âmbito do Modelo eesc
% ---
\usepackage{lipsum}				       % para geração de dummy text
\usepackage[printonlyused]{acronym}
\usepackage[table]{xcolor}
% ---


% ---
% Informações de dados para CAPA e FOLHA DE ROSTO
% ---
%
% Título:
%	1. Título em português
%	2. Título em inglês
\titulo{Proposta do projeto final de Data Science - Violência Armada e suas características e incidências  nos EUA.}{
Proposal of the final Data Science project - Armed Violence and its future trends in the USA}
%
% Autor:
%	1. Nome completo do autor
%	2. Formato de nome para bibliografia
\author{
  Lerisson Florêncio de Freitas\\
  \texttt{lff3@cin.ufpe.br}
  \and\\
  Matheus Raz de Oliveira Leandro\\
  \texttt{mrol@cin.ufpe.br}}
%
% Cidade
\local{Recife}
% Ano de defesa
\data{2018}
%Área de concentração da pesquisa
\areaconcentracao{Ciência da Computação}
% Nome do orientador IF1015 Intro a Ciência dos Dados
%Disciplina sobre a Ciência da Informação do Centro de Informática da UFPE
\orientador{Dr. Renato Vimieiro}
%
% compila o indice
% ---
\makeindex
% ---

% ---
% Compila a lista de abreviaturas e siglas
% ---
\makenomenclature
% ---

% ---
% Inserir ficha catalográfica
%
% Caso o comando \inserirfichacatalografica seja definido, a %ficha catalográfica
% será inserida atrás da folha de rosto. Caso contrário a página será deixada em
% branco.
%
% CUIDADO: Esta opção deve ser preenchida antes do comando \maketitle
% ---
%entre em contato com a biblioteca para obter a sua ficha catalográfica em arquivo pdf. Essa %folha só será inserida no documento após a sua defesa.

%\inserirfichacatalografica{fichaCatalografica.pdf}
% ---

% ---
% Inserir folha de aprovação
%
% Caso o comando \inserirfolhaaprovacao seja definido, a a folha de aprovação
% será inserida. Além disso, conforme Resolução CoPGr 5890, as informações 
% de rodapé são inseridas apropriadamente na folha de rosto.
%
% CUIDADO: Esta opção deve ser preenchida antes do comando \maketitle
% ---
% baseie-se no modelo desse documento e gere a sua folha de %rosto em arquivo pdf.

%\inserirfolhaaprovacao{folhaAprovacao.pdf}
% ---

% ----
% Início do documento
% ----

\begin{document}

% ----------------------------------------------------------
% ELEMENTOS PRÉ-TEXTUAIS
% ----------------------------------------------------------
\pretextual

% ---
% Insere Capa, Folha de rosto, Ficha catalográfica (se inserida)
% e folha de aprovação (se inserida).
% ---
\maketitle



% ---
% RESUMO e ABSTRACT
% ---

% Resumo em português - as palavras entre chaves são as palavras-chave do trbalho
\begin{comment}
\begin{resumo}{Latex. Abntex. Normas USP}

 	 Segundo a \citeonline[3.1-3.2]{NBR6028:2003}, o resumo deve ressaltar o  objetivo, o método, os resultados e as conclusões do documento. A ordem e a extensão  destes itens dependem do tipo de resumo (informativo ou indicativo) e do  tratamento que cada item recebe no documento original. O resumo deve ser  precedido da referência do documento, com exceção do resumo inserido no  próprio documento. (\ldots) As palavras-chave devem figurar logo abaixo do  resumo, antecedidas da expressão Palavras-chave:, separadas entre si por  ponto e finalizadas também por ponto.
     
    Para auxiliá-lo com o latex, o Apêndice 
  \ref{cap_exemplos} apresenta os resultados dos comandos incluídos no arquivo ape\_comandos/abntex2-modelo-include-comandos.tex 

\end{resumo}
\end{comment}

% inserir lista de ilustrações
% ---
%\listailustracoes
% ---

% ---
% inserir lista de tabelas
% ---
%\listatabelas
% ---

% ---
% inserir lista de abreviaturas e siglas
% ---
%\listasiglas{abrev/Abreviaturas}
% ---

% ---
% inserir o sumario
% ---
%\sumario
% ---

% ----------------------------------------------------------
% ELEMENTOS TEXTUAIS
% ----------------------------------------------------------
\mainmatter

% ----------------------------------------------------------
% Introdução
% ----------------------------------------------------------
%\include{cap_introducao/introducao}


% ----------------------------------------------------------
% Fundamentação
% ----------------------------------------------------------
%\include{cap_fundamentacao/fundamentacao}

% ----------------------------------------------------------
% Proposta de pesquisa
% ----------------------------------------------------------
\chapter[Proposta do projeto]{Proposta}

\par
\textbf{
\section{Base}
} Nossa base de estudo foi coletada no repositório online Kaggle para aprendizagem de máquina: \href{https://www.kaggle.com/jameslko/gun-violence-data}{Dados de violência armada. Registro abrangente de mais de 260 mil incidentes de violência armada nos EUA entre 2013-2018}
\section{Informações sobre a base
}
Registrou-se mais de 260 mil incidentes de violência armada, com informações detalhadas sobre cada incidente, disponíveis no formato CSV. O arquivo CSV contém dados de todos os incidentes registrados de violência armada nos EUA entre janeiro de 2013 e março de 2018, inclusive.
\\
\\
\textbf{Coleta dos dados:}
\begin{itemize}
\item \textbf{Fase 1:} para cada data entre 1/1/2013 e 31/03/2018, um script Python consultou todos os incidentes que ocorreram naquela data em particular, depois digitalizou os dados e os escreveu em um arquivo CSV. Cada mês tem seu próprio arquivo CSV, com exceção de 2013, já que não foram registrados muitos incidentes a partir de então.
\item \textbf{Fase 2:} cada entrada foi aumentada com dados adicionais que não podem ser visualizados diretamente na página de resultados da consulta, como informações do participante, dados de geolocalização etc.
\item \textbf{Fase 3:} as entradas foram classificadas em ordem crescente e depois mescladas em um único arquivo CSV.
\end{itemize}

\section{
Proposta} 
\par
Atualmente, faltam quantidades grandes e facilmente acessíveis de dados detalhados sobre a violência armada em geral. 
Para tal resolvemos focar na análise do contexto Norte Americano por ser um país que permite a regulamentação do uso de armas em alguns estados, gerando assim um índice de violência que demanda uma maior incidência de crimes cometidos com o uso delas. 
\par
Como mostrado na descrição da base estudada, relatórios como este e este mostram que Nikolas Cruz exibiu muitos sinais de alerta nas redes sociais antes do tiroteio, por exemplo. 
Para esse fim, a partir dessa base de dados, podemos identificar e levantar hipóteses que nos levem a entender melhor a necessidade de um reforço em tais áreas com maior índice de ocorrências desse viés, tais quais:
\\
\\
\textbf{Hipóteses:}

\begin{enumerate}
\item Em que cidades ou condados têm maior probabilidade de ocorrer um crime à mão armada?
\item Qual faixa etária de pessoas está presente na participação de tais crimes?
\item Que tipo de armas possuem maior incidência nos casos de violência?
\item Relacionar em quais  distritos de votação para senadores ocorrem mais violência armada.
\end{enumerate}



% ----------------------------------------------------------
% Experimentos e avaliação dos resultados
% ----------------------------------------------------------

%\include{cap_experimentos/experimentos}

% ---
% Finaliza a parte no bookmark do PDF, para que se inicie o bookmark na raiz
% ---
\bookmarksetup{startatroot}% 
% ---

% ---
% Conclusão
% ---
%\include{cap_conclusao/conclusao}


 

% ----------------------------------------------------------
% ELEMENTOS PÓS-TEXTUAIS
% ----------------------------------------------------------
\postextual

% ----------------------------------------------------------
% Referências bibliográficas
% ----------------------------------------------------------
%\bibliography{bib/abntex2-modelo-references}

% ----------------------------------------------------------
% Glossário
% ----------------------------------------------------------
%
%\glossary

% ----------------------------------------------------------
% Apêndices
% ----------------------------------------------------------
% ---
% Inicia os apêndices
% ---
\begin{comment}

\begin{apendicesenv}
% Imprime uma página indicando o início dos apêndices
\partapendices
% ----------------------------------------------------------
% Incluir Apêndice
% ----------------------------------------------------------
% ----------------------------------------------------------
% Capitulo com exemplos de comandos inseridos de arquivo externo 
% ----------------------------------------------------------
\include{ape_comandos/abntex2-modelo-include-comandos}

\end{apendicesenv}
% ---

% ----------------------------------------------------------
% Anexos
% ----------------------------------------------------------
% ---
% Inicia os anexos
% ---
\begin{anexosenv}
% Imprime uma página indicando o início dos anexos
\partanexos
% ---
% Incluir Anexo
% ---
\chapter{Morbi ultrices rutrum lorem.}

%o comando lipsum[] comando serve apenas para incluir texto no documento para efeito de visualização do formato.
\lipsum[1-25]
\section{Test}
\lipsum[1-20]


\end{anexosenv}
\end{comment}

\end{document}