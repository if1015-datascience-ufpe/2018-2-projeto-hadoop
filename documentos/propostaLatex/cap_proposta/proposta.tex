\chapter[Proposta do projeto]{Proposta}

\par
\textbf{
\section{Base}
} Nossa base de estudo foi coletada no repositório online Kaggle para aprendizagem de máquina: \href{https://www.kaggle.com/jameslko/gun-violence-data}{Dados de violência armada. Registro abrangente de mais de 260 mil incidentes de violência armada nos EUA entre 2013-2018}
\section{Informações sobre a base
}
Registrou-se mais de 260 mil incidentes de violência armada, com informações detalhadas sobre cada incidente, disponíveis no formato CSV. O arquivo CSV contém dados de todos os incidentes registrados de violência armada nos EUA entre janeiro de 2013 e março de 2018, inclusive.
\\
\\
\textbf{Coleta dos dados:}
\begin{itemize}
\item \textbf{Fase 1:} para cada data entre 1/1/2013 e 31/03/2018, um script Python consultou todos os incidentes que ocorreram naquela data em particular, depois digitalizou os dados e os escreveu em um arquivo CSV. Cada mês tem seu próprio arquivo CSV, com exceção de 2013, já que não foram registrados muitos incidentes a partir de então.
\item \textbf{Fase 2:} cada entrada foi aumentada com dados adicionais que não podem ser visualizados diretamente na página de resultados da consulta, como informações do participante, dados de geolocalização etc.
\item \textbf{Fase 3:} as entradas foram classificadas em ordem crescente e depois mescladas em um único arquivo CSV.
\end{itemize}

\section{
Proposta} 
\par
Atualmente, faltam quantidades grandes e facilmente acessíveis de dados detalhados sobre a violência armada em geral. 
Para tal resolvemos focar na análise do contexto Norte Americano por ser um país que permite a regulamentação do uso de armas em alguns estados, gerando assim um índice de violência que demanda uma maior incidência de crimes cometidos com o uso delas. Com isso, o nosso objetivo é traçar perfis de crimes que tenham a presença do uso de armas de fogo.
\section{
Metodologia} 
\par
Podemos identificar e levantar padrões que nos levem a entender melhor a necessidade de um reforço em tais áreas com maior índice de ocorrências desse viés avaliando correlações entre os atributos presentes no \textit{dataset}, onde aplicamos a análise exploratória sobre os dados para corroborar hipóteses levantadas por nós, tais quais:
\\
\\

\textbf{Hipóteses:}

\begin{enumerate}
\item Em que cidades ou condados têm maior probabilidade de ocorrer um crime à mão armada?
\item Qual faixa etária de pessoas está presente na participação de tais crimes?
\item Que tipo de armas possuem maior incidência nos casos de violência?
\item Relacionar em quais  distritos de votação para senadores ocorrem mais violência armada.
\end{enumerate}
\\

\par
No que diz respeito a análise exploratória será montadas a visualização destes dados  através do que foi aprendido na disciplina com uso de gráficos que carregam consigo relações de medidas estatísticas, para verificar e entender o comportamento dos dados.  Tudo isso com o objetivo de obter a validação ou a refutação das hipóteses anteriores. Podemos afirmar que nessa etapa procuraremos descobrir relações como:
\\
\begin{enumerate}
\item A distribuição de crimes a mão armada por grupos de pessoas no país.
\item A evolução desses crimes nos condados no decorrer dos anos 
\item A relação de Renda por incidência de crimes armados.
\item Cruzar a relação acima pela a idade das pessoas que cometem esses crimes.
\end{enumerate}
\\
\par
Também será construído um mapa de calor para inferir o que não foi coberto pela a análise por se só dos dados. Assim ficará clara a distribuição desses crimes no EUA.    
\par
Após as análises descritas acima, será aplicada, também, a técnica de Árvore de Decisão com o objetivo de extrair alguns padrões nessa base de dados, e predizer descrições para \textit{clusters} dos crimes coletados.
